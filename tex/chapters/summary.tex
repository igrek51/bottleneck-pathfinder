\chapter{Podsumowanie}
\label{ch:podsumowanie}

Większość popularnych algorytmów wykorzystywanych do planowania tras dla wielu robotów mobilnych (agentów) opiera się o A*.

Kooperacyjne planowanie tras jest ogólną techniką koordynacji dróg wielu jednostek.
Znajduje zastosowanie, gdzie wiele jednostek może komunikować się ze sobą, przekazując informację o ich ścieżkach.
Poprzez planowanie wprzód w czasie, jak również i w przestrzeni, jednostki potrafią schodzić sobie z drogi nawzajem w celu uniknięcia kolizji.
Metody kooperacyjnego planowania są bardziej skuteczne i znajdują trasy wyższej jakości niż te uzyskane przez A* z metodą Local Repair.

Wiele z udoskonaleń przestrzennego algorytmu A* może być również zaadaptowane do czasprzestrzennego A*.
Ponadto, wprowadzenie wymiaru czasu otwiera nowe możliwości do rozowju algorytmów znajdowania dróg.

Najbardziej obiecującym algorytmem wydaje się być metoda WHCA*.
Algorytmy takie jak WHCA*, HCA* lub CA* wprowadzają dyskretyzację czasu.
Aby wydajnie prowadzić obliczenia, zakłada się, że każdy ruch robota trwa tyle samo. 
Wprowadza to upraszczające, błędne założenie, że ruch robota na pole w kierunku poziomym lub pionowym trwa tyle samo, co na ukos.

W wielu przypadkach metody do planowania bezkolizyjnych tras w systemach wieloagentowych mogą być wykorzystywane zamiennie zarówno do wyznaczania trajektorii robotów mobilnych, jak i w grach komputerowych, np. strategiach czasu rzeczywistego do planowania tras wielu jednostek.

Zaprezentowane algorymy mogą znaleźć zostosowanie również w środowiskach z ciągłą przestrzenią oraz w dynamicznych środowiskach, jeśli ścieżki będą przeliczane po wykryciu zmiany na mapie.
