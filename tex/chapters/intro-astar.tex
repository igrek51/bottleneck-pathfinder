\chapter{Algorytmy oparte na A*}
\label{ch:astar}

Kiedy pojedynczy agent dokonuje znalezienia drogi do wyznaczonego celu, prosty algorytm A* sprawdza się bardzo dobrze. Jednak w przypadku, gdy wiele agentów porusza się w tym samym czasie, to podejście może się nie sprawdzić, powodując wzajemne blokowanie się i zakleszczenie jednostek. Rozwiązaniem tego problemu może być kooperacyjne znajdowanie tras. Jednostki będą mogły skutecznie przemieszczać się przez mapę omijając trasy wyznaczone przez inne jednostki oraz schodząc innym jednostkom z drogi, jeśli to konieczne. \cite{cooppath}

Zagadnienie znajdowania drogi jest ważnym elementem sztucznej inteligencji zaimplementowanej w wielu grach komputerowych. Chociaż klasyczny algorytm A* potrafi doprowadzić pojedynczego agenta do celu, to jednak dla wielu agentów wymagane jest zastosowanie innego podejścia w celu unikania kolizji. Algorytm A* może zostać zaadaptowany do replanowania trasy na żądanie, w przypadku wykrycia kolizji tras (Local Repair A* lub D*). Jednak takie podejście nie jest zadowalające pod wieloma względami. Na trudnych mapach z wieloma wąskimi korytarzami i wieloma agentami może to prowadzić do zakleszczenia agentów w wąskich gardłach lub do cyklicznego zapętlenia ruchu agentów. \cite{cooppath}

\section{Algorytm A*}
$TODO$

\section{Local Repair A*}
W algorytmie Local Repair A* (LRA*) każdy z agentów znajduje drogę do celu, używając algorytmu A*, ignorując pozostałe roboty oprócz ich obecnych sąsiadów. Roboty zaczynają podążać wyznaczonymi ścieżkami do momentu, aż kolizja z innym robotem jest nieuchronna. Wtedy następuje ponowne przeliczenie drogi pozostałej do przebycia, z uwzględnieniem nowo napotkanej przeszkody.

Możliwe (i całkiem powszechne \cite{cooppath}) jest uzyskanie cykli (tych samych sekwencji ruchów powtarzających się w nieskończoność), dlatego zazwyczaj wprowadzane są pewne modyfikacje, aby rozwiązać ten problem. Jedną z możliwości jest zwiększanie wpływu losowego szumu na wartość heurystyki. Kiedy agenci zachowują się bardziej losowo, prawdopodobne jest, że wydostaną się z problematycznego położenia i spróbują podążać innymi ścieżkami.

Algorytm ten ma jednak sporo poważnych wad, które szczególnie ujawniają się w trudnych środowiskach z dużą liczbą przeszkód. Wydostanie się z zatłoczonego wąskiego gardła może trwać bardzo długo. Prowadzi to również do ponownego przeliczania trasy w prawie każdym kroku. Wciąż możliwe jest również odwiedzanie tych samych lokalizacji w wyniku zapętleń.

\section{Algorytm D*}
D* ({\it Dynamic A* Search}) jest przyrostowym algorytmem przeszukiwania. Jest modyfikacją algorytmu A* pozwalającą na szybsze replanowanie trasy w wyniku zmiany otoczenia (np. zajmowania wolnego pola przez innego robota). Wykorzystywany jest m.in. w nawigacji robota do określonego celu w nieznanym terenie. Początkowo robot planuje drogę na podstawie pewnych założeń (np. nieznany teren nie zawiera przeszkód). Podążając wyznaczoną ścieżką, robot odkrywa rzeczywisty stan mapy i jeśli to konieczne, wykonuje ponowne planowanie trasy na podstawie nowych informacji.
Często wykorzystywaną implementacją (z uwagi na zoptymalizowaną złożoność obliczeniową) jest algorytm {\it D* Lite} \cite{dstarlite}.

\section{Cooperative A*}

\section{Hierarchical Cooperative A*}

\section{Windowed Hierarchical Cooperative A*}

$TODO$
