\section{Metoda pól potencjałowych}
\label{ch:potential-fields}

W aplikacji zaimplementowano także metodę pól potencjałowych.
Cykl symulacji powtarzany jest w stałych cyklach zegarowych.
30 FPS
W każdym takim cyklu wyznaczana jest siła wypadkowa działająca na robota.
Wpływa to na zmianę wektora prędkości robota, a w konsekwencji na zmianę jego położenia.

Zastosowano działania na wektorach.

Siła "przyciągająca" robota do punktu docelowego ma stałą wartość, aby w każdej odległości od celu robot podążał do niego tak samo.

Wartość sumarycznej siły pochodzącej od wszystkich przeszkód jest ograniczana do maksymalnej wartości, aby wpływ od wielu przeszkód nie był na tyle znaczący, aby uniemożliwić robotowi dotarcie do celu.


zalety: real time - potrzeba mało obliczeń, szybka metoda

Problem minimów lokalnch
bardzo słaba skuteczność
siła z pięciu punktów przeszkody, a i tak efekt wallhacka
może się zderzyć z przeszkodą w wyniku nadmiernego rozpędzenia i braku wyhamowania, z powodu rozpatrywania jako punkty nie jako bryły