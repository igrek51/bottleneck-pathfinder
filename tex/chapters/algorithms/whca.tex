\section{Implementacja algorytmu WHCA*}
\label{ch:alg-whca}

$TODO$ rozwiązuje bottleneck - nawet radzi sobie dla większej liczby.
Roboty wykazują inteligentne zachowanie, potrafią kooperować ze sobą, schodzić sobie z drogi, nawet, gdy już dotarły do swojego celu

WHCA - własny, schemat blokowy, pseudokod
wiele trzeba było zrobić samemu, bo nic nie było podane
zaprojektowany własny algorytm
planowanie tylko części ścieżki (długości okna)
koszt pozostania w miejscu - niezerowy,  chyba, że w celu (żeby zachęcić do szybszego dotarcia)
po przejsciu algorytmu, wyłanianie najlepszego rozwiązania, 2 kryteria

rozszerzona wersja o dynamiczne przydzielanie priorytetów i rozszerzanie okna
wada - nie moze się dwóch agentów "cofać". MOże tylko uciekać  z drogi ważniesjzemu