\section{Ograniczenia}
W aplikacji zostały wprowadzone pewne uproszczenia, które m.in. przyspieszają i ułatwiają proces obliczeniowy planowania trajektorii dla robotów.
\begin{enumerate}
	\item Założono, że ruch ukośny robota o jedno pole (po przekątnej) trwa tyle samo, co ruch poziomy lub pionowy. Wprowadzono takie przybliżenie ze względu na możliwość ujednolicenia jednostki wymiaru czasu w tablicy rezerwacji pól przez agentów.
	\item Nie uwzględniono czasu obrotu robota podczas zmiany kierunku jazdy. Założono, że czas ten jest zerowy, gdyż celem zaprojektowanego algorytmu było skupienie się na rozwiązaniu innego zagadnienia - problemu zakleszczeń w wąskich gardłach.
\end{enumerate}
