\section{Dynamiczny przydział priorytetów}
\label{ch:alg-priorities-allocation}

Obok algorytmu WHCA* zrealizowano własną metodę dynamicznego przydziału priorytetów oraz skalowania okna czasowego.
Układ priorytetów w metodzie WHCA* ma znaczący wpływ na wynik planowania tras. Od priorytetów robotów zależy kolejność wyznaczania tras.
Jest to na tyle istotne, gdyż podczas wyznaczania tras uwzględniane są tylko roboty o priorytetach wyższych (których planowanie odbyło się wcześniej).

Podobnie jak w algorytmie LRA*, również w metodzie WHCA* należy wykrywać kolizje (odpowiednio wcześnie).
Nawet w przypadku, gdy roboty podążają wzdłuż zaplanowanych ścieżek, wciiąż mogą zdarzyć się kolizje.
Może to wystąpić w przypadku, gdy robot, który planuje drogę wczesniej (ma wyższy priorytet), nie uwzględnia, że następnemu robotowi może nie udać się wyznaczenie trajektorii (z powodu właśnie zaplanowanej trasy).



% TODO screen z przypadku kolizjii

Zaproponowana meotda polega na zwiększaniu priorytetów robotów w przypadku niepowodzenia w znalezieniu trasy.


zwiększanie priorytetu w każdym kroku, jeśli nie znaleziono trasy, z próbą ponownego znalezienia w następnym kroku
zwiększanie okna czasowego do max z priorytetów


rozszerzona wersja o dynamiczne przydzielanie priorytetów i rozszerzanie okna

przed planowaniem: sortowanie po priorytetach
brak znalezienia ścieżki (kolizje z innymi robotami i za niski priorytet ) - promocja priorytetu - zwiększenie o 1 
resetCollidedRobots po planowaniu (podobnie jak w LRA*)
rozmiar okna czasowego zwiększa się, jeśli któryś z robotów ma wyższy priorytet

przykłady rozwiązywanych problemów
