\chapter{Oprogramowanie symulacyjne}
\label{ch:simulation-app}

Na potrzeby pracy zostało stworzone oprogramowanie symulacyjne, które posłużyło do przeprowadzenia testów skuteczności algorytmów planowania tras oraz wizualizacji działania metod.
W tym rozdziale opisano techniczne rozwiązania wykorzystane podczas tworzenia oprogramowania.
W aplikacji zostały zaimplementowane algorytmy planowania tras opisane w rozdziale \ref{ch:alg-impl}.
Prezentacja ich działania odbywa się poprzez wizualizację ruchu robotów mobilnych w czasie rzeczywistym. 

\section{Funkcjonalności aplikacji}
Aplikacja umożliwia dowolne definiowanie przez użytkownika środowiska, w którym poruszają się roboty. Obejmuje to:
\begin{itemize}
	\item wybór rozmiaru mapy - dowolną wysokość oraz szerokość. Mapa nie musi być kwadratowa.
	\item możliwość wygenerowania mapy za pomocą generatora labiryntów (por. \ref{ch:mazegen}) lub manualnego umieszczania przeszkód na mapie za pomocą myszki,
	\item wybór liczby robotów i dokonanie ich automatycznego rozmieszczenia na mapie (w losowych polach z pominięciem pól zajętych). Użytkownik ma także możliwość manualnego dodawania i usuwania robotów.
\end{itemize}

Aplikacja przeprowadza symulację ruchu robotów w czasie rzeczywistym. W oprogramowaniu zostały zaimplementowane trzy algorytmy planowania ruchu robotów. Są to:
\begin{itemize}
	\item Metoda pól potencjałowych (por. \ref{ch:theory-potential-fields});
	\item Local-Repair A* (por. \ref{ch:alg-collision-avoid});
	\item WHCA*3 - Windowed Hierarchical Cooperative A* z dynamicznym przydziałem priorytetów oraz skalowaniem okna czasowego (por. \ref{ch:alg-whca}, \ref{ch:alg-priorities-allocation}).
\end{itemize}
Wizualizacja każdego z tych algorytmów dostępna jest na osobnej zakładce w aplikacji.
