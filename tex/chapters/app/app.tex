\chapter{Oprogramowanie symulacyjne}
\label{ch:simulation-app}

Na potrzeby pracy zostało stworzone oprogramowanie symulacyjne, które posłużyło do przeprowadzenia testów algorytmów oraz wizualizacji ich działania.
Prezentacja działania algorytmów planowania tras możliwa jest dzięki wizualizacji ruchu robotów odbywajacego się w czasie rzeczywistym. 

\section{Funkcjonalności aplikacji}

Aplikacja umożliwia dowolne definiowanie przez użytkownika środowiska, w którym poruszają się roboty. Obejmuje to:
\begin{itemize}
	\item wybór rozmiaru mapy - dowolna wysokość oraz szerokość. Mapa nie musi być kwadratowa.
	\item możliwość wygenerowania mapy za pomocą generatora labiryntów (por. \ref{ch:mazegen}) lub manualnego umieszczania przeszkód na mapie za pomocą myszki,
	\item wybór liczby robotów i dokonanie ich automatycznego rozmieszczenia na mapie (w losowych polach z pominięciem pól zajętych). Użytkownik ma także możliwość manualnego dodawania i usuwania robotów.
\end{itemize}
położenia przeszkód 
Aplikacja umożliwia symulację ruchu robotów oraz definiowanie położenia przeszkód przez użytkownika.

Zakres pracy - Implementacja oprogramowania symulacyjnego


\section{Interfejs użytkownika}
$TODO$ opis UI, zaznaczanie dróg, celów, płynne animacje, osobny wątek do UI
kółka to roboty, wyświetlanie ścieżek, celów
co robi każdy przycisk

\section{Stack technologiczny}
$TODO$ Wykorzystane technologie i narzędzia - opis technologii:
Java 8 - lambda, functional interfaces, streamy,
Java FX - FXML, Spring (core): IoC, DI; Spring Boot, testy jednostkowe jUnit, git, IntelliJ Ultimate, Maven, Linux, logback, Guava - joiner

\section{Testy}
$TODO$ TDD - Test driven development, testy jednostkowe do algorytmów pathfinding

\section{Struktura aplikacji}
$TODO$ lista beanów / serwisów, struktura widok, prezenter, kontroler; osobny wątek w tle do obliczeń + synchronizacja, wątek UI - zapewnienie REal-time, prawie MVP


\section{Screeny}
$TODO$ screeny

\section{Zaimplementowane metody}
$TODO$ potential fields, A*, LRA*, WHCA*

\section{featurey}
ustawianie random seeda
ponowne wykonanie symulacji - te same warunki
przełaczanie między metodami na zakładkach ?
wykonanie symulacji w pojedynczych krokach
resizable window - responsive

\section{Ograniczenia}
$TODO$ nałożone uproszczenia: ruch skośny trwa tyle samo, czas dyskretny, brak czasu na obrót

$TODO$ publikacja na GitHub, licencja MIT, filmiki na YT