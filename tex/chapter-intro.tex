\chapter{Wprowadzenie} % (fold)
\label{cha:wprowadzenie}

TODO: Celem pracy jest pierdolenie o Szopenie oraz pierdolenie kotka za pomocą młotka.


$TODO$ Konspekt:

Cel pracy
	system dla szpitali, dostarczania dokumentów, paczek
Założenia
	Aplikacja powinna umożliwiać symulację ruchu robotów oraz definiowanie położenia przeszkód przez użytkownika
	Planowanie tras dotyczy robotów holonomicznych
Wstęp teoretyczny
	algorytm A*
	systemy wilorobotowe
	wyznaczanie bezkolizyjnych tras
	słowniczek:
		robot holonomiczny
	tłumaczenie prezentacji o priorytetyzacji
	tłumaczenie innych prac angielskich :)
Projekt algorytmu wyznaczania trajektorii dla pojedynczego robota
	algorytm A* - opis algorytmu, implementacja, znajdzie rozwiązanie jeśli istnieje, nie jest optymalny, modyfikacja o omijanie ścian, ale z możliwością ruchu ukośnego
	potential field - słabe, problem minimów lokalnych bardzo mocny
Algorytm detekcji i zapobiegania kolizjom między robotami
	wyznaczanie globalnie optymalnego rozwiazania (centralnie) - jaki algorytm?
	lokalne podejmowanie decyzji na podstawie zasad (ruch drogowy)
	priorytetyzacja i wyznaczanie dróg:
		A star time-space - algorytm ?
		Path coordination - algorytm ?
		metoda wyznaczania priorytetów:
			metoda Monte Carlo?
			metoda największego spadku?
Implementacja oprogramowania symulacyjnego
	wykrzystane technologie - czyli co tygryski lubią najbardziej
	Java, Spring, Spring Boot, JavaFX, IntelliJ Ultimate, wzorce projektowe, klasy Immutable
	kluczowe algorytmy: symulacja ruchu robota w A* - schemat blokowy
	umożliwienie tworzenia mapki, definiowanie położenia przeszkód przez usera
	diagram klas?
	screeny
Przeprowadzenie testów symulacyjnych
	dużo testów
	dużo screenów
