\noindent {\Large\textbf{Streszczenie}}\\

\noindent {\Large\textbf{Planowanie bezkolizyjnych tras dla zespołu robotów mobilnych}}\\

% interlinia 0
\begin{singlespacing}

Przedmiotem niniejszej pracy jest przegląd, implementacja i wykonanie testów wybranych metod planowania bezkolizyjnych tras dla systemów wielorobotowych.
W ramach pracy zostały opracowane własne metody wyznaczania dróg dla wielu robotów mobilnych na dwuwymiarowych mapach.
Metody te znajdują zastosowanie w środowiskach z dużą liczbą przeszkód i wąskimi gardłami, gdzie często występuje problem zakleszczania się robotów.

% wstęp teoretyczny - przegląd metod
W pierwszych rozdziałach dokonano przeglądu i analizy najczęściej wykorzystywanych podejść do kooperacyjnego planowania tras.
Wskazano również na podobieństwo występowania podobnego zagadnienia w grach komputerowych.

% algorytmy
W kolejnej części omówiono algorytmy opracowane na potrzeby oprogramowania symulacyjnego.
Efektem pracy jest m.in. własna implementacja algorytmu {\it Windowed Hierarchical Cooperative A*} rozszerzonego o dodatkowe procedury dynamicznego przydziału priorytetów oraz zmiennego okna czasowego.

% oprogramowanie symulacyjne
W ramach pracy wykonano oprogramowanie pozwalające na symulację oraz wizualizację metod planowania tras w systemach wielorobotowych.
Opracowana aplikacja desktopowa realizuje założone funkcjonalności.
Użytkownik ma możliwość dowolnego definiowania środowiska a wizualizacja ruchu robotów mobilnych odbywa się w czasie rzeczywistym.
W jednym z rozdziałów zostały omówione techniczne rozwiązania wykorzystane podczas tworzenia aplikacji.

% testy
Stworzone oprogramowanie symulacyjne posłużyło również do przeprowadzenia testów skuteczności algorytmów planowania tras.
W kolejnej części przedstawiono wyniki automatycznie przeprowadzonych testów zaimplementowanych metod planowania.
Wykonanie licznych eksperymentów pozwoliło stwierdzić, że opracowana własna metoda uzyskała największą skuteczność w wyznaczaniu bezkolizyjnych tras i prowadzeniu agentów do ich celów w porównaniu do pozostałych wariantów.
Na potrzeby wszystkich testów efektywności oraz wydajności badanych metod przeprowadzono łącznie $76\ 800$ automatycznie zarządzanych symulacji ruchu robotów.
Testy te były przeprowadzane w losowo wygenerowanych środowiskach, w różnych warunkach.
W celu generalizacji uzyskanych wyników posłużono się statystyką.


Zauważalny brak ogólnie dostępnych, wydajnych i efektywnych algorytmów kooperacyjnego planowania dróg świadczy o potrzebie rozwoju i rozpowszechnienia metod tego typu, co także było celem tej pracy i zostało zrealizowane poprzez opublikowanie kodów źródłowych opracowanej aplikacji na portalu dla projektów typu {\it open-source}.

Na podstawie obserwacji zachowań robotów sterowanych przez opracowaną metodę planowania tras można niejednokrotnie stwierdzić, że roboty te potrafią wykazywać się "inteligentną" koordynacją ruchu.
Przystosowując się do zmiennych warunków, roboty potrafią niekiedy podejmować akcje, które z pozoru nie byłyby korzystne dla indywidualnego agenta (np. oddalają go od celu), jednak pozwalają one osiągnąć wspólny cel dzięki koooperacji.

\flushbottom
\textbf{\\Słowa kluczowe: }planowanie tras, systemy wielorobotowe
\end{singlespacing}
