\noindent {\Large\textbf{Streszczenie}}\\

\noindent {\Large\textbf{Planowanie bezkolizyjnych tras dla zespołu robotów mobilnych}}\\

% interlinia 0
\begin{singlespacing}

$TODO$ uzupełnić
% wyciągnąć z podsumowania i z intro
% w pracy zawarto szczegółową implementację, aby rozpowszechnić użycie algorytmów planowania (zauważalny brak)

% Zakres pracy:
% \begin{itemize}
% 	\item Projekt algorytmu wyznaczania trajektorii dla pojedynczego robota
% 	\item Algorytm detekcji i zapobiegania kolizjom między robotami
% 	\item Implementacja oprogramowania symulacyjnego
% 	\item Przeprowadzenie testów symulacyjnych
% \end{itemize}

% Podstawowe wymagania:
% \begin{itemize}
% 	\item Aplikacja powinna umożliwiać symulację ruchu robotów oraz definiowanie położenia przeszkód przez użytkownika.
% 	\item Planowanie tras dotyczy robotów holonomicznych.
% \end{itemize}


\flushbottom
\textbf{\\Słowa kluczowe: }planowanie tras, systemy wielorobotowe
\end{singlespacing}
