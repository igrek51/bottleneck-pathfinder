\noindent {\Large\textbf{Streszczenie}}\\

\noindent {\Large\textbf{Planowanie bezkolizyjnych tras dla zespołu robotów mobilnych}}\\

% interlinia 0
\begin{singlespacing}

Przedmiotem niniejszej pracy jest przegląd, implementacja i wykonanie testów wybranych metod planowania bezkolizyjnych tras w systemach wielorobotowych.
W ramach pracy zostały opracowane własne metody wyznaczania dróg dla wielu robotów mobilnych na dwuwymiarowych mapach.
Metody te znajdują zastosowanie w środowiskach z dużą liczbą przeszkód i wąskimi gardłami, gdzie często występuje problem zakleszczania robotów.

% wstęp teoretyczny - przegląd metod
W pierwszych rozdziałach dokonano przeglądu i analizy najczęściej wykorzystywanych podejść do kooperacyjnego planowania tras.

% algorytmy
W kolejnej części omówiono algorytmy opracowane na potrzeby oprogramowania symulacyjnego.
Efektem pracy jest m.in. własna implementacja algorytmu {\it Windowed Hierarchical Cooperative A*} rozszerzonego o dodatkowe procedury dynamicznego przydziału priorytetów oraz zmiennego okna czasowego.

% oprogramowanie symulacyjne
Efektem pracy jest oprogramowanie pozwalające na symulację oraz wizualizację metod planowania tras w systemach wielorobotowych.
Opracowana aplikacja realizuje założone funkcjonalności.
W następnym rozdziale zostały omówione techniczne rozwiązania wykorzystane podczas jej tworzenia.
Wizualizacja ruchu robotów mobilnych odbywa się w czasie rzeczywistym.

% testy
W kolejnej części przedstawiono wyniki automatycznie przeprowadzonych testów zaimplementowanych metod planowania.
Stworzone oprogramowanie symulacyjne posłużyło również do przeprowadzenia testów skuteczności algorytmów planowania tras.
Wykonanie obszernych testów automatycznie zarządzanych symulacji pozwoliło stwierdzić, że opracowana własna metoda uzyskała największą skuteczność w wyznaczaniu bezkolizyjnych tras i prowadzeniu agentów do ich celów w porównaniu do pozostałych metod.
Na potrzeby wszystkich testów efektywności oraz wydajności badanych metod przeprowadzono łącznie $76\ 800$ automatycznie zarządzanych symulacji ruchu robotów.
Testy zostały przeprowadzone w losowo wygenerowanych środowiskach, następnie posłużono się statystyką w celu generalizacji uzyskanych wyników.


Zauważalny brak ogólnie dostępnych, wydajnych i efektywnych algorytmów kooperacyjnego planowania dróg świadczy o potrzebie rozwoju i rozpowszechnienia metod tego typu, co także było celem pracy i zostało zrealizowane poprzez opublikowanie kodów źródłowych opracowanej aplikacji na portalu dla projektów typu {\it open-source}.

Na podstawie obserwacji zachowań robotów sterowanych przez opracowaną metodę planowania tras można stwierdzić, że roboty te niekiedy potrafią wykazywać się "inteligentną" koordynacją ruchu.
Przystosowując się do zmiennych warunków, roboty potrafią podejmować akcje, które z pozoru nie byłyby korzystne dla indywidualnego agenta (np. oddalają go od celu), jednak pozwalają one osiągnąć wspólny cel dzięki koooperacji.

\flushbottom
\textbf{\\Słowa kluczowe: }planowanie tras, systemy wielorobotowe
\end{singlespacing}
