\section{Konspekt pracy}
\label{ch:konspekt}
\begin{itemize}
	\item (P) Cel i zakres pracy, założenia, zastosowanie
	\item (P) Podobieństwo do gier RTS
	\item (P) Wstęp teoretyczny:
	\begin{itemize}
		\item (P) Podstawowe pojęcia
		\item (P) Cooperative Pathfinding, przegląd metod planowania tras, rozproszone i priorytetyzowane planowanie
		\item (P) Metoda pól potencjałowych
		\item (P) Metoda Path Coordination
		\item (P) Algorytm A* - szczegółowo, algorytm, rola funkcji heurystycznej
		\item (P) Local Repair A*, D*
		\item (P) Cooperative A* - wprowadzenie trzeciego wymiaru, tablica rezerwacji
		\item (P) Hierarchical Cooperative A*
		\item (P) Windowed Hierarchical Cooperative A*
	\end{itemize}
	\item Generowanie mapy - labiryntu do testów: własny algorytm, teoria grafów, własności mapy
	\item Time-space A* - pseudokod, schemat blokowy, własne heurystyki i modyfikacje
	\item Metoda przydziału / zmiany priorytetów - zwiększanie i rekalkulacja
	\item Metoda pól potencjałowych - minima lokalne
	\item Implementacja aplikacji
	\begin{itemize}
		\item stack technologiczny: Java 8, Java FX, Spring, Spring Boot, testy jednostkowe jUnit, git, IntelliJ, Maven, Linux / Windows
		\item screeny
		\item schemat struktury klas aplikacji
	\end{itemize}
	\item Obszerne testy - porównanie wyników metod (WHCA*, CA*, LRA*?) przy tych samych warunkacj początkowych, porównanie czasu wykonania od rozmiaru okna czasowego
	\item Ograniczenia - nałożone uproszczenia: ruch skośny, czas dyskretny, brak czasu na obrót
\end{itemize}