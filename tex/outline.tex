\chapter{Karta tematu}
\label{ch:topic}

Karta tematu: \\
Temat pracy: Planowanie bezkolizyjnych tras dla zespołu robotów mobilnych \\
Temat pracy (w jęz. ang.): Path planning for a group of mobile robots

Zakres pracy:
\begin{itemize}
	\item Projekt algorytmu wyznaczania trajektorii dla pojedynczego robota
	\item Algorytm detekcji i zapobiegania kolizjom między robotami
	\item Implementacja oprogramowania symulacyjnego
	\item Przeprowadzenie testów symulacyjnych
\end{itemize}

Podstawowe wymagania:
\begin{itemize}
	\item Aplikacja powinna umożliwiać symulację ruchu robotów oraz definiowanie położenia przeszkód przez użytkownika.
	\item Planowanie tras dotyczy robotów holonomicznych.
\end{itemize}

Założenia:
• The environment is deterministic. However, agents may not move if they are trying to move to an obstacle or another agent.
• For simplicity, agents only have the path finding task. Moreover, the agents won’t disappear when they reach their target. They just stay at the target block and the block cannot be used by other agents.
• No two agents can have the same target.
• Agents have a full view of the map and other agents.
• Actions are: lewo, prawo, góra, dół i na ukosy w labiryncie

\chapter{Konspekt pracy}
\label{ch:konspekt}
\begin{itemize}
	\item (P) Cel i zakres pracy, założenia, zastosowanie
	\item (P) Podobieństwo do gier RTS
	\item (P) Wstęp teoretyczny:
	\begin{itemize}
		\item (P) Podstawowe pojęcia
		\item (P) Cooperative Pathfinding, przegląd metod planowania tras, rozproszone i priorytetyzowane planowanie
		\item (P) Metoda pól potencjałowych
		\item (P) Metoda Path Coordination
		\item (P) Algorytm A* - szczegółowo, algorytm, rola funkcji heurystycznej
		\item (P) Local Repair A*, D*
		\item (P) Cooperative A* - wprowadzenie trzeciego wymiaru, tablica rezerwacji
		\item (P) Hierarchical Cooperative A*
		\item (P) Windowed Hierarchical Cooperative A*
	\end{itemize}
	\item Generowanie mapy - labiryntu do testów: własny algorytm, teoria grafów, własności mapy
	\item Time-space A* - pseudokod, schemat blokowy, własne heurystyki i modyfikacje
	\item Metoda przydziału / zmiany priorytetów - zwiększanie i rekalkulacja
	\item Metoda pól potencjałowych - minima lokalne
	\item Implementacja aplikacji
	\begin{itemize}
		\item stack technologiczny: Java 8, Java FX, Spring, Spring Boot, testy jednostkowe jUnit, git, IntelliJ, Maven, Linux / Windows
		\item screeny
		\item schemat struktury klas aplikacji
	\end{itemize}
	\item Obszerne testy - porównanie wyników metod (WHCA*, CA*, LRA*?) przy tych samych warunkacj początkowych, porównanie czasu wykonania od rozmiaru okna czasowego
	\item Ograniczenia - nałożone uproszczenia: ruch skośny, czas dyskretny, brak czasu na obrót
\end{itemize}