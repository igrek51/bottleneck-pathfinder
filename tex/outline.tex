\chapter{Karta tematu}
\label{ch:topic}

Karta tematu: \\
Temat pracy: Planowanie bezkolizyjnych tras dla zespołu robotów mobilnych \\
Temat pracy (w jęz. ang.): Path planning for a group of mobile robots

Zakres pracy:
\begin{itemize}
	\item Projekt algorytmu wyznaczania trajektorii dla pojedynczego robota
	\item Algorytm detekcji i zapobiegania kolizjom między robotami
	\item Implementacja oprogramowania symulacyjnego
	\item Przeprowadzenie testów symulacyjnych
\end{itemize}

Podstawowe wymagania:
\begin{itemize}
	\item Aplikacja powinna umożliwiać symulację ruchu robotów oraz definiowanie położenia przeszkód przez użytkownika.
	\item Planowanie tras dotyczy robotów holonomicznych.
\end{itemize}

Założenia:
• The environment is deterministic. However, agents may not move if they are trying to move to an obstacle or another agent.
• For simplicity, agents only have the path finding task. Moreover, the agents won’t disappear when they reach their target. They just stay at the target block and the block cannot be used by other agents.
• No two agents can have the same target.
• Agents have a full view of the map and other agents.
• Actions are: lewo, prawo, góra, dół i na ukosy w labiryncie

\chapter{Konspekt pracy}
\label{ch:konspekt}
\begin{itemize}
	\item Wstęp teoretyczny:
	\begin{itemize}
		\item Cooperative Pathfinding
		\item algorytm A* - szczegółowo
		\item przegląd metod planowania tras dla wielu robotów
		\item artykuł o Cooperative Pathfinding, time-space A*
		\item artykuł o wyznaczaniu priorytetów i metodach planowania tras (prezentacja): Path Coordination, time-space A*
		\item metoda ładunków - problem minimów lokalnych
		\item replanowanie po wykruciu kolizcji (algorytm D*)
		\item algorytmy WHCA* i IADPP
		\item Reciprocal Collision Avoidance
		\item metody przydziału priorytetów - zwiększanie i przeliczanie
		\item metody zcentralizowane vs rozproszone (porównanie)
		\item time-space A*, heurystyki, Reservation Table
	\end{itemize}
	\item zastosowanie: ciasne korytarze, częsty problem kolizji, szpitale, transport dokumentów, paczek
	\item generowanie mapy - labiryntu do testów: własny algorytm, teoria grafów, własności mapy
	\item time-space A* - pseudokod, schemat blokowy, własne heurystyki, modyfikacje
	\item metoda przydziału / zmiany priorytetów
	\item ograniczenia - nałożone uproszczenia: ruch skośny, czas dyskretny, brak czasu na obrót
	\item Implementacja aplikacji - stack technologiczny: Java 8, Java FX, Spring, Spring Boot, testy jednostkowe jUnit, git, IntelliJ, Maven, Linux / Windows; schemat klas aplikacji
	\item obszerne testy, porównanie wyników metod przy tych samych warunkacj początkowych
\end{itemize}