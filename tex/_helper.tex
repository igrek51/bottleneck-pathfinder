% Latex syntax helper

% rozdziały, podrozdziały
\chapter{Rozdział}
\label{ch:rozdzial}

\section{Sekcja}
\subsection{Podsekcja}
\subsubsection{Podpodsekcja}

% wymuszenie podziału strony
\clearpage

% text formatting
cytat z bib \cite{siemiatkowska}.
link do rys / rozdziału (por. \ref{ch:dstar}).
{\it italic}
{\bf bold}

% wypunktowanie
\begin{itemize}
	\item poz1
\end{itemize}

\vspace{-1em}
\begin{itemize}[noitemsep]
	\item poz1 bez odstepów
\end{itemize}

\begin{enumerate}
	\item pierwszy
\end{enumerate}

% obrazki
\begin{figure}
	\centering
	\includegraphics[width=1.0\columnwidth]{img/}
	\caption{Opis-wstaw gdziekolwiek. Źródło: gra}
	\label{fig:img_}
\end{figure}

\begin{figure}[H]
	\centering
	\includegraphics[width=1.0\columnwidth]{img/}
	\caption{Opis-wstaw TUTAJ. Źródło: gra}
	\label{fig:img_}
\end{figure}

% wiele obrazków w kolumnach
\begin{figure}
	\centering
		\subfloat[]{\includegraphics[width=0.4\columnwidth]{img/}}
		\qquad
		\subfloat[]{\includegraphics[width=0.4\columnwidth]{img/}}
	\label{fig:img_}
	\caption{Opis.
	(a) 
	(b) }
\end{figure}

% Równania
$equation$

\begin{gather}
 	f(n) = h(n) = \sqrt{(x_n - x_g)^2 + (y_n - y_g)^2} h^*(n) (x_g, y_g)
 	\label{eq_astar} 
\end{gather}
 gdzie:

 $g(n)$ - zmienna12

 $h(n)$ - zmienna3

% Definicja
\begin{definition}{\bf Pojęcie\\}
	Definicja
\end{definition}

% Algorytm - pseudokod
\begin{algorithm}[H]
  \caption{Opis}\label{alg:ddd}
  \begin{algorithmic}[1]
\REQUIRE {$w$ - szerokość}
\STATE $x \gets poleZ.x$
\STATE {\bf function} wyburzDrogę($mapa$, $poleZ$, $poleDo$)
	\begin{ALC@g}
	\WHILE{$x < poleDo.x$}
		\STATE $mapa[x\verb-++-][y] \gets WOLNE$ 
	\ENDWHILE
	\end{ALC@g}
\STATE {\bf end function}
	\end{algorithmic}
\end{algorithm}
