% Latex syntax helper

% rozdziały, podrozdziały
\chapter{Rozdział}
\label{ch:rozdzial}

\section{Sekcja}
\subsection{Podsekcja}
\subsubsection{Podpodsekcja}

% wymuszenie podziału strony
\clearpage

% text formatting
cytat z bib \cite{siemiatkowska}.
link do rys / rozdziału (por. \ref{ch:dstar}).
{\it italic}
{\bf bold}

% wypunktowanie
\begin{itemize}
	\item poz1
\end{itemize}

\vspace{-1em}
\begin{itemize}[noitemsep]
	\item poz1 bez odstepów
\end{itemize}

\begin{enumerate}
	\item pierwszy
\end{enumerate}

% obrazki
\begin{figure}
	\centering
	\includegraphics[width=0.8\columnwidth]{img/}
	\caption{Opis-wstaw gdziekolwiek. Źródło: gra}
	\label{fig:name}
\end{figure}

\begin{figure}[H]
	\centering
	\includegraphics[width=0.8\columnwidth]{img/}
	\caption{Opis-wstaw TUTAJ. Źródło: gra}
	\label{fig:name}
\end{figure}

% wiele obrazków w kolumnach
\begin{figure}
	\centering
		\subfloat[]{\includegraphics[width=0.4\columnwidth]{img/}}
		\qquad
		\subfloat[]{\includegraphics[width=0.4\columnwidth]{img/}}
	\caption{Opis.
	(a) 
	(b) }
	\label{fig:name}
\end{figure}

% Równania
$equation$

\begin{gather}
 	f(n) = h(n) = \sqrt{(x_n - x_g)^2 + (y_n - y_g)^2} h^*(n) (x_g, y_g)
 	\label{eq_astar} 
\end{gather}
 gdzie:

 $g(n)$ - zmienna12

 $h(n)$ - zmienna3

% Definicja
\begin{definition}{\bf Pojęcie\\}
	Definicja
\end{definition}

% Algorytm - pseudokod
\begin{algorithm}[H]
	\caption{Opis}\label{alg:name}
  \begin{algorithmic}[1]
\Function{funkcja}{$mapa$, $start$, $cel$}
	\State $closed \gets \varnothing$  \Comment{pusta lista zamkniętych}
	\For{$wezel \in mapa$}
		\State $wezel.g = \infty$ \Comment{domyślnie nieskończony koszt - odległość od startu}
	\EndFor
	\If{$mapa[cel.x][cel.y] == ZABLOKOWANE$} \Comment{Punkt docelowy zablokowany}
		\State \Return $\varnothing$ \Comment{Brak rozwiązania}
	\EndIf
	\While{$open \ne \varnothing$} \Comment{dopóki lista otwartych nie jest pusta}
		\State $obecny \gets $ \Call{znajdźMinF}{$open$} \Comment{Szukamy pola o najniższej wartości f}
	\EndWhile
	\State \Return $\varnothing$ \Comment{Przeanalizowano wszystkie węzły, brak istniejącej ścieżki}
\EndFunction
  \end{algorithmic}
\end{algorithm}

% Tabelka
\begin{table}
\caption{oppis} \label{tab:name} 
\centering
\begin{tabular}{| l | c | r |}
\hline
1 & a & A \\ \hline
2 & b & B \\ \hline
\end{tabular}
\end{table}
