\chapter{Podsumowanie} % (fold)
\label{cha:podsumowanie}

Praca miała na celu projekt systemu do monitorowania zużycia mediów komunalnych.
Projektowany system miał pozwalać na wizualizację bieżących i historycznych stanów oraz powiadamiać o zdarzeniach niepożądanych takich jak braki odpowiedzi urządzeń sieciowych, co może świadczyć o ich awarii, lub o znaczących różnicach w stosunku do przewidywanego dobowego zużycia medium.

W trakcie pracy powstało oprogramowanie UMonit wspierające liczniki komunikujące się za pomocą protokołu M-Bus lub pozwalające, za pomocą innych narzędzi, na integrację z tym protokołem.
Projekt UMonit składał się z dwóch podprojektów UMonitApp oraz UMonitWeb.
UMonitApp zajmuje się odczytywanie liczników zgodnie z normą PN-EN 13757, zapisem danych pomiarowych oraz wykrywaniem błędów.
UMonitWeb jest narzędziem do przeglądania zebranych przez UMonitApp danych w formie listy ostatnich znanych wartości parametrów danego licznika lub wykresów zmiany tych parametrów.
Pozwala również na wylistowanie zdarzeń, które pojawiły się w systemie.

System ten może być z powodzeniem stosowany do celów rozliczeniowych, gdzie sześciominutowy interwał odczytów jest dopuszczalny.
Jest on łatwo integrowalny z infrastrukturą domową, gdzie używane są liczniki posiadające wyjścia impulsowe lub moduły komunikacyjne M-Bus.
Podłączenie do domowego routera pozwala na dostęp do danych z dowolnego komputera, laptopa, telefonu czy tabletu będących w sieci LAN tworzonej przez ten router.
Obsługa 250 urządzeń pozwala na zastosowanie rozwiązania w blokach mieszkalnych do monitorowania poszczególnych pięter lub całych budynków.

Ze względu dużą wartość interwału odczytów UMonit nie nadaje się do zastosowań przemysłowych, gdzie częstotliwości sprawdzania stanu parametrów powinna być wielokrotnie większa.

Błędy mogą pojawić się zarówno na etapie implementacji, jak i na etapie modelowania czy tworzenia założeń systemu.
Pierwsze mogą być trudne do wykrycia, a drugie kosztowne w rozwiązaniu.
Podczas implementacji pojawiło się wiele nieprzewidzianych problemów, które należało rozwiązać, aby móc kontynuować prace nad projektem.

Jednym z nich było rozgłaszanie komend \textit{echo} oraz \textit{status} w wielu interfejsach sieciowych.
Problem pojawił się, kiedy Odroid C1+ był połączony z konwerterem ETH2 za pomocą kabla ethernetowego oraz z lokalną siecią WiFi za pomocą zewnętrznej karty sieciowej.
Okazało się, że system Ubuntu 16.04 nie propagował rozgłoszeń typu $ 255.255.255.255 $ na wiele interfejsów, ponieważ urządzenie posiadało więcej niż jeden adres IP.
Rozwiązaniem było pobranie listy dostępnych interfejsów sieciowych i wysłanie komendy \textit{echo} lub \textit{status} na adres rozgłoszeniowy tego interfejsu.
Pomocna w uzyskaniu tych parametrów okazała się biblioteka Qt udostępniająca odpowiednie do tego klasy.

Kolejnym problemem było niespodziewane odrzucanie kilku kolejnych ramek.
Działo się to raz na kilkanaście odczytów.
Problem miał dwa źródła.
Pierwszym było założenie, że odbieranie i wstępne przetwarzanie danych może odbywać się niezależnie od procesu wysyłania danych.
Oznaczało to tyle, że bufory na dane nie były przygotowywane na odbiór nowych ramek.
Należało je wyczyścić przy każdym żądaniu.
Drugim źródłem była źle ustawiona wartość timeoutu, która powodowała zbyt wczesną ponowną próbę odczytania danych, co prowadziło do przerywania przez licznik poprzedniej transmisji.

UMonit nie jest jeszcze gotowym produktem mogącym być zainstalowanym w domu.
Wymaga jeszcze wiele testów kompatybilności z licznikami innych mediów oraz urządzeniami innych producentów.
W przyszłości system może być rozszerzony lub połączony z istniejącym systemem rachunkowym obliczającym wielkości opłat za zużywane media.
Notyfikacje mogłyby otrzymać różne priorytety określane automatycznie lub definiowane przez użytkownika.
UMonit może również pozwalać w przyszłości na definiowanie nowych alarmów przez użytkownika, który definiowałby także sposób powiadamiania o ich zajściu np. za pomocą maila lub SMS-a dla najbardziej krytycznych.

Do systemu mogłoby zostać dodane wsparcie dla protokołu zdalnego odczytu WM-Bus pozwalającego na zbieranie danych bez ingerencji w strukturę mieszkania w celu przeciągnięcia przewodów komunikacyjnych dla magistrali M-Bus.
Dodatkowo wprowadzić można system uwierzytelniania użytkowników w celu zabezpieczenia i ograniczenia dostępu do danych osobom niepowołanym.

Cel pracy został osiągnięty, a z założenia tworzonego systemu w pełni zrealizowane.
% chapter wnioski (end)